\thispagestyle{plain}
\sisetup{range-phrase=-}

\section*{Kurzfassung}
In dieser Arbeit wird ein Vergleich zweier Vetoregionen für den IceCube Detektor im Energiebereich von $\SIrange{1}{1000}{GeV}$ durchgeführt.
Die Vetoregionen (DC und EXT) unterscheiden sich in ihrer Grundfläche.
Auf Basis von simulierten Ereignissen atmosphärischer Myonen, Elektron- und Myon-Neutrinos werden durch Anwenden der Vetos zwei Datensätze generiert. 
Nach vorheriger Attributsselektion durch mRMRe werden diese in einer zweistufigen Separation mit Random Forests separiert. 
Es werden zuerst alle Neutrinoereignisse von Myonereignissen getrennt.
Von den übrigen Neutrinoereignissen werden Myon-Neutrinos als Signal selektiert.
Nach jeder Separation werden die relativen Differenzen der Ereignisraten der beiden Vetos und die Korrelation zwischen rekonstruierter und wahrer Energie der Ereignisse verglichen.
Mit einer erwarteten Myon-Neutrino-Rate von $\SI{4.63}{µHz}$ und einer Korrelation von 0.7 zeigt sich das DC-Veto dem EXT-Veto überlegen.

\section*{Abstract}
\begin{english}
    Within this thesis a comparison between two veto regions (DC and EXT) for the IceCube detector is made in the $\SIrange{1}{1000}{GeV}$ energy range.
    Simulation data consisting of muons, electron neutrinos and muon neutrinos is used.
    Applying the vetoes to those events yields two new sets of events, each consisting only of events passing the corresponding veto.
    Relevant features are selected via mRMRe and used in a two staged random forest separation.
    First, all neutrinos are separated from muons, then muon neutrino candidates are separated from the remaining electron neutrinos.
    For each stage the passing event rates and correlation between reconstructed and true energy are compared.
    The DC-Veto is shown to be superior, with an expected muon neutrino rate of $\SI{4.63}{µHz}$ and a correlation of 0.7.
    %(Whereas the EXT-Veto has a rate of $\SI{1.67}{µHz}$ and a correlation of 0.67.?)
\end{english}
