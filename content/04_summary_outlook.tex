\chapter{Zusammenfassung und Ausblick}
\subsection*{Zusammenfassung}
In dieser Arbeit werden zwei gleich aufgebaute zweischrittige Separationen zur Selektion von atmosphärischen Myon-Neutrino-Kandidaten durchgeführt und ihre Ergebnisse verglichen.
Die dafür notwendigen Ereignisse entstehen jeweils durch Filtern mit unterschiedlichen Vetoregionen, dem DC- und dem EXT-Veto, die alle Ereignisse verwerfen, deren Interaktion vermutlich im vom Veto definierten Vetovolumen stattfindet.
Verwendet werden dafür Simulationsdaten von Myon, Elektron-Neutrino und Myon-Neutrino Ereignissen mit Attributen wie Energie- und Richtungsrekonstruktionen.
Zuerst wird die jeweilige Vetobedingung auf alle vorhandenen Ereignisse angewendet.
Dann werden mittels mRMRe 44 Attribute für die Separation von Myon- und Neutrinoereignissen ausgewählt.
In der Neutrinoseparation wird in einer fünffachen Kreuzvalidierung ein Random Forest mit 50 Bäumen auf allen vorhandenen Myonereignissen und jeweils $\num{250000}$ Elektron- sowie Myon-Neutrino-Ereignissen trainiert.
Dieser wird auf alle Simulationsdaten angewendet.
Alle Ereignisse mit einer Konfidenz unter 0.9 werden dem Untergrund zugeordnet und verworfen, alle anderen sind Signalereigniskandidaten. 
Für das DC-Veto bleiben nach diesem Schritt $\num{54077}$ simulierte Ereignisse mit einer Gesamtrate von $\SI{40.76}{µHz}$ übrig, für das EXT-Veto bleiben $\num{10716}$ Ereignisse mit einer Rate von $\SI{8.01}{µHz}$ übrig.
Unter den selektierten Ereignissen sind nur noch Elektron- und Myon-Neutrinos. 
Sie werden in einer zweiten Separation voneinander getrennt.
Dafür werden mittels mRMRe 36 Attribute selektiert und in einer fünffachen Kreuzvalidierung je ein Random Forest mit 200 Bäumen trainiert.
Die Ereignisse werden so in Myon-Neutrino- (Signal) und Elektron-Neutrino-Kandidaten (Untergrund) separiert.
Der Konfidenzschnitt wird bei 0.84 gesetzt, was beim DC-Veto zu einer Signalreinheit von $\SI{95.3(8)}{\%}$ führt. 
Für das DC-Veto bleiben Signalereigniskandidaten mit einer Gesamtrate von $\SI{4.63}{µHz}$ übrig, für das EXT-Veto $\SI{1.67}{µHz}$.
Auf Basis der Raten an verbleibenden Signalereigniskandidaten wird ein Vergleich der beiden Vetoregionen durchgeführt.
Nach dem ersten Separationsschritt werden die Neutrinoereignisraten zusammen mit der Konfidenz und wahren Energie der Ereignisse verglichen.
Es zeigt sich, dass das DC-Veto für Konfidenzen größer 0.64 in jedem Energiebereich über höhere Ereignisraten als das EXT-Veto verfügt.
Beim Vergleich der rekonstruierten und wahren Energie der Ereignisse ergibt sich für die Vetos eine Korrelation von $ρ_\text{DC} = 0.7$ und $ρ_\text{EXT} = 0.67$, sowie ein Überschätzen der wahren Energie. 
Danach werden die Raten von Kandidaten für Myon-Neutrino-Ereignisse nach der zweiten Separation verglichen.
Die Raten des DC-Vetos sind für jeden Konfidenz- und Energiebereich höher als die des EXT-Vetos.


\subsection*{Ausblick}

Weiterführend kann die in dieser Arbeit aufgestellte Separations- und Analysekette automatisiert werden, um weitere Vetodefinitionen und ihre Effekte auf die Qualität der Separation zu testen.
Zudem sollte ein Vergleich der Simulationsdaten mit experimentellen Daten durchgeführt werden und Attribute mit zu geringer Übereinstimmung entfernt werden.
Damit wird sichergestellt, dass die Separation auf experimentellen Daten ähnliche Ergebnisse liefert, wie in der Kreuzvalidierung.
Zusätzlich kann die durchgeführte Separation weiter verbessert werden.
Um die Trennkraft zu erhöhen können neue, auf den Niederenergiebereich angepasste, Attribute erstellt werden.
Anstatt die Vetoentscheidungen wie bisher als Filter für Ereignisse zu verwenden, können sie dem Lerner als Attribut übergeben werden.

Bei der Neutrinoselektion gehen die meisten Signalereignisse verloren.
Hauptursache dafür ist wahrscheinlich die geringe Anzahl an \texttt{Corsika} Untergrund\-ereignissen.
Um dem entgegen zu wirken können mehr Myonereignisse simuliert und zum Trainieren verwendet werden.
Es können auch experimentelle Daten als Myonuntergrund verwendet werden.
Möglich ist dies durch das hohe Signal-Untergrund-Verhältnis von $1$:$10^3$.
Der geringe Anteil an Neutrinos in solch einem Untergrunddatensatz beeinträchtigt die Separationsqualität des Random Forests durch seine Robustheit gegenüber falsch gelabelten Ereignissen nur in geringem Maße.
Um Stabilität und Trennkraft der Separation zu erhöhen kann eine höhere Zahl von Bäumen trainiert werden.
Auf Grund von Limitationen der Rechenzeit waren die in dieser Arbeit durchgeführten Neutrinoseparationen auf 50 Bäume begrenzt.
Die Separation der Myon-Neutrinos sollte sich durch eine mehrfache Separation für unterschiedliche Energiebereiche verbessern lassen.
Eine so optimierte Separationskette kann auf aktuelle experimentelle Daten angewendet werden, um etwa eine Bestimmung des Myon-Neutrinoflusses zu ermöglichen. 
