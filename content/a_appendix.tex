\chapter{Anhang}


\section{Verwendete Software}
Die Vorprozessierung der im HDF5~\cite{hdf5} Format vorliegenden Daten wurde mit verschiedenen Python~\cite{pythonFirst} Paketen durchführt.
H5py~\cite{h5py} zum Einlesen und erneuten Schreiben der Daten.
Numpy~\cite{numpy} und Pandas~\cite{pandas} zum Anwenden von mathematischen Funktionen als auch zum Transformieren der Daten.
Die Attributsselektion wurde mit der R~\cite{R} Bibliothek mRMRe~\cite{mrmre} in Python über den R zu Python wrapper rpy2~\cite{rpy2} durchgeführt.
Die Separation wurde mit dem WEKA Random Forest~\cite{Weka2009} in RapidMiner~\cite{RapidMiner} durchgeführt. 
Alle Grafiken wurden mit Matplotlib~\cite{matplotlib} erstellt.

\section{NaN-Verteilung}
\label{nanverteilung}

\begin{figure}
\begin{center}
    \includegraphics{nanratio}
\end{center}
\vspace{-2em}
    \caption{Histogramm der NaN-Anteile.}
\label{fig:nanverteilung}
\end{figure}

\section{Vergleich von gewichteten und ungewichteten Korrelationskoeffizienten}
\label{korrkoeff}

Die Ereignisse sind je nach ihrer Energie unterschiedlich gewichtet.
Weiter kann nicht ausgeschlossen werden, dass die einzelnen Attribute für verschiedene Energiebereiche unterschiedlich stark miteinander oder mit dem Zielattribut korreliert sind.
Diese sich verändernden Korrelationen würden für eine Verfälschung der Attributsauswahl sorgen.
Daher wird untersucht, ob die Gewichtung der Ereignisse die Korrelationskoeffizienten beeinflusst.
Zum Vergleich werden die gewichteten und ungewichteten Pearson-Korrelationskoeffizienten der Attribute verwendet.

In Abbildungen~\ref{fig:korrelationen} und~\ref{fig:crosscorrelation} ist zu sehen, dass die relativen Abweichungen der beiden Arten von Koeffizienten nur gering sind.
Die gewichteten und ungewichteten Koeffizienten sind also weitestgehend äquivalent.

\begin{figure}
\begin{center}
    \includegraphics{comparison/correlation_energy}
\end{center}
\vspace{-2em}
\caption{Histogramm der relativen Differenzen zwischen gewichteten und ungewichteten Pearson-Korrelationskoeffizienten der Attribute mit der wahren Energie.}
\label{fig:korrelationen}
\end{figure}
\begin{figure}
\begin{center}
    \includegraphics{comparison/crosscorrelation}
\end{center}
\vspace{-2em}
\caption{Histogramm der relativen Differenzen zwischen gewichteten und ungewichteten Pearson-Korrelationskoeffizienten aller möglichen Attributskombinationen.}
\label{fig:crosscorrelation}
\end{figure}




\section{Vergleich von mRMR Methoden anhand von Testseparationen}
Verglichen wird die in Abschnitten~\ref{neutrinomrmr} und~\ref{myonmrmr} verwendete mRMRe Variante und die mRMR der RapidMiner Feature Selection Extension~\cite{FeatureSelection} mit dem MID Kriterium (weiter FSE-mRMR genannt).
Dafür werden fünffach kreuzvalidierte Separationen von Elektron- und Myon-Neutrinos mit einem Random Forest durchgeführt.
Bei beiden Separationsketten unterscheiden sich nur die Arten der Attributsselektionen.
Die gemittelten relativen Abweichungen der Reinheiten und Effizienzen sind in Abbildung~\ref{fig:mrmremid} dargestellt.
Positive Werte stehen für eine höhere Qualität der Separation auf Basis der Attributsmenge der mRMRe Methode im Vergleich zur FSE-mRMR.
Sie sind bis zu Konfidenzwerten von $\num{0.4}$ mit Null verträglich.
Danach weichen sie um bis zu 5\% voneinander ab, die Standardabweichungen werden teilweise so groß wie die relativen Abweichungen selbst.
Die relative Reinheit der mRMRe ist höher und ihre relative Effizienz niedriger.
Die mRMRe Methode liefert also vergleichbare Ergebnisse wie die FSE-mRMR Varianten. 
\begin{figure}
\begin{center}
    \includegraphics{comparison/mrmre_mid}
\end{center}
\vspace{-2em}
\caption{Aufgetragen sind die gemittelten relativen Differenzen der Qualitätsparameter Effizienz und Reinheit zweier Separationen mit vorheriger Attributsselektion durch mRMRe und FSE-mRMR. Positive Werte  stehen für eine bessere Separation mit der mRMRe Attributsmenge. Fehlerbalken sind die Standardabweichungen der Mittelwerte.}
\label{fig:mrmremid}
\end{figure}



\section{Separationsqualität der EXT-Veto Neutrinoselektion}
\label{extvetoAnhang}
Dieses Kapitel beinhaltet der Vollständigkeit halber Grafiken für das EXT-Veto, die in Abschnitt~\ref{neutrinoseparation} und~\ref{myonseparation} für das DC-Veto dargestellt wurden.
\begin{figure}
\begin{center}
    \includegraphics{separation/EXTconfhistNu}
\end{center}
\vspace{-2em}
\caption{Aus der Neutrinoseparation mit dem EXT-Veto. Histogramme der Konfidenzverteilungen der gemittelten Ereignisraten von Myonen (Untergrund) und Neutrinos (Signal) in Hertz. Die Standardabweichungen aus der fünffachen Kreuzvalidierung sind als Fehlerbalken dargestellt. }
\end{figure}
\begin{figure}
\begin{center}
    \includegraphics{separation/EXTqualityNUENUMU}
\end{center}
\vspace{-2em}
\caption{Aus der Neutrinoseparation mit dem EXT-Veto. Aufgetragen sind die gemittelten Qualitätsparameter Reinheit (rot) und Effizienz (blau) mit ihren Standardabweichungen gegen die Konfidenz.}
\end{figure}
\begin{figure}
\begin{center}
    \includegraphics{separation/EXTconfhistNUENUMU}
\end{center}
\vspace{-2em}
\caption{Aus der Myon-Neutrinoseparation mit dem EXT-Veto. Histogramme der Konfidenzverteilungen der gemittelten Ereignisraten von Myonen (Untergrund) und Neutrinos (Signal) in Hertz. Die Standardabweichungen aus der fünffachen Kreuzvalidierung sind als Fehlerbalken dargestellt. }
\end{figure}
\begin{figure}
\begin{center}
    \includegraphics{separation/EXTqualityNu}
\end{center}
\vspace{-2em}
\caption{Aus der Myon-Neutrinoseparation mit dem EXT-Veto. Aufgetragen sind die gemittelten Qualitätsparameter Reinheit (rot) und Effizienz (blau) mit ihren Standardabweichungen gegen die Konfidenz.}
\end{figure}

\begin{figure}
\begin{center}
    \includegraphics{/comparison/rates/eSigExtNu}
\end{center}
\vspace{-2em}
    \caption{Aufgetragen ist wahre Energie gegen rekonstruierte Energie nach dem Konfidenzschnitt bei 0.9 für das EXT-Veto. Die absoluten Ereignisraten sind farblich kodiert.}
\label{fig:eNUEXT}
\end{figure}

