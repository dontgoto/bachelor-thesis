\chapter{Einleitung}
\sisetup{range-phrase=-}

In der Astroteilchenphysik werden entfernte Objekte im Weltall mit Teleskopen beobachtet, um über ausgesandte Teilchen, der so genannten kosmischen Strahlung, Informationen zu gewinnen.
Die kosmische Strahlung besteht aus Photonen, sowie geladenen und ungeladenen Teilchen.
Mit den gewonnenen Informationen ist es möglich Rückschlüsse auf ihre Quellen und Produktionsmechanismen zu ziehen.
Neutrinos sind ein Teil der ungeladenen kosmischen Strahlung und können mit dem IceCube~\cite{icecube-detector-reference} Teleskop am Südpol detektiert werden. 
Da sie nur schwach wechselwirken werden sie indirekt durch das Tscherenkowlicht von dabei entstehenden geladenen Teilchen gemessen.
Der geladene Teil der kosmischen Strahlung produziert bei Interaktionen in der Atmosphäre Neutrinos und kann somit indirekt untersucht werden.
Von diesen so genannten atmosphärischen Neutrinos sind Myon-Neutrinos die zahlreichsten. 
In dieser Analyse sind sie Ziel der Untersuchungen und werden fortan als Signal bezeichnet.
Auf Grund des geringen Wirkungsquerschnitts von Neutrinos mit Materie sind diese Signalereignisse im Vergleich zu anderen Ereignissen wie atmosphärischen Myonen, dem dominierenden Untergrund, sehr selten.
Um einen hoch reinen Datensatz an Signalereignissen zu erhalten, ist es notwendig diese vom Untergrund zu separieren.
Eine wichtige Vorstufe dafür sind Vetos.
Mit ortsbasierten Vetos werden Teilchen verworfen, deren Wechselwirkung wahrscheinlich nicht im Detektionsvolumen stattgefunden hat.  
Diese Entscheidung erfolgt anhand des geschätzten Ladungsschwerpunkts und weiteren Informationen der Ereignisse.
Vor allem im betrachteten Energiebereich, dem Niederenergiebereich von IceCube, sind solche Vetos besonders wichtig.
Nach theoretischen Flussmodellen sind die Neutrinoraten hier im Verhältnis zu den Untergrundraten besonders klein. 
DeepCore eignet sich besonders zur Messung dieser Neutrinos, da er dichter instrumentiert ist und diese Ereignisse genauer auflösen kann.

Ist die Wahl des Detektionsvolumens größer als DeepCore ist auf Grund der geringeren Dichte an Sensoren zu erwarten, dass die rekonstruierte Energie der Ereignisse weniger stark mit ihrer wahren Energie korreliert.
Es ist auch zu erwarten, dass bei einem größeren Vetovolumen mehr Untergrundereignisse verworfen werden, während die Rate an detektierten Signalereignissen kleiner wird. 
In Hinsicht auf diese wesentlichen Unterschiede werden zwei Vetoregionen DC und EXT, von denen EXT das größere Detektionsvolumen hat, verglichen.
Dies geschieht auf Basis von simulierten Myon-, sowie Myon-Neutrino- und Elektron-Neutrino-Ereignissen.
Da auf experimentellen Daten der Ereignistyp nicht bekannt ist, müssen zunächst Myon-Neutrinos von den restlichen Ereignissen separiert werden. 
Von den separierten Ereignissen werden die erwähnten Korrelationen und Raten verglichen.

